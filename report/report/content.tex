%--------------------------------------------------------------------------
% COMP61011, project
% Edward John
% University of Manchester
%--------------------------------------------------------------------------
\definecolor{light-gray}{gray}{0.9}
\newcommand{\todo}[1]{
	\colorbox{light-gray}{
		\parbox{\linewidth}{
			\textbf{TODO} #1
		}
	}
}

\nocite{github}

%--------------------------------------------------------------------------
% Abstract
%--------------------------------------------------------------------------
\begin{abstract} 
Issue tracking in software development usually requires the submission of text explaining the issue. Issues tend to have labels
associated with them to help categorise them so that they can be more easily searched for. Examples of labels could include
\textit{bug}, \textit{suggestion} or \textit{question} and may not be mutually exclusive. GitHub is a website that provides
issue tracking to public code repositories. This allows any GitHub user to submit issues where repository collaborators can then
label them.

This projects explores how common labels in particular (\textit{bug}, \textit{suggestion} and \textit{question}) can be
predicted for GitHub issues by first extracting features from the issue text in the form of words and phrases. This can then be
processed using machine learning techniques such as the k-Nearest Neighbour classifier and finally measuring and comparing the
performance over several different feature selection techniques to improve the accuracy of the classifier.
\end{abstract}

%--------------------------------------------------------------------------
% Introduction
%--------------------------------------------------------------------------
\section{Introduction}
GitHub is a website of the US company GitHub, Inc. that offers online hosting of code repositories and support for project team
collaboration. One of the provided features for each hosted "Git" repository is an issue tracker. The issue tracker allows users
to submit issues. Types of issues can include; bug reports, feature suggestions, improvements or questions. When an issue is
submitted, collaborators of the repository can add labels to it to help categorise the issue. Labels may not mutually exclusive,
therefore an issue could have the labels; bug and high priority. Because labels have to be manually assigned by collaborators,
issues may not be labelled for some time depending on the availability and number of collaborators. As a result filtering and
sorting may not select or collate all the related issues of a particular type, and collaborators have to be relied upon to be
diligent in keeping up to date with their labelling new issues.

Similar to spam e-mail classifiers, an issue can be classified by analysing the words and phrases in the issue body text. Doing
this requires selecting the right word and phrases (features) to look for. This helps remove noise such as stop words which are
found in nearly all bodies of text for anything and improves the performance of the algorithm by reducing the search space.
Feature selection can also be used to improve the classifiers individually for the labels \textit{bug}, \textit{suggestion} and
\textit{question} by giving each feature a different weight counting towards each label classification.

%--------------------------------------------------------------------------
% Background
%--------------------------------------------------------------------------
\section{Background}
\subsection{Information extraction}
Before any experiments could be made on predicting labels for issues, information extraction was performed by writing an
application that downloaded all the issues for a given GitHub repository and extracted features and labels from each issue. The
application used the Octokit \cite{octokit} library provided by GitHub to download the issues using their application
programming interface (API). These features consisted of words and pairs of words (phrases) from within the body text of an
issue. The labels were individual non-mutual exclusive flags for whether the issue had been labelled as \textit{bug},
\textit{suggestion} or \textit{question}.

\subsection{Word / phrase frequencies}
As a preliminary form of feature selection, a word and phrase frequency analysis was performed on all the issues for three
different GitHub repositories (jquery-handsontable, TypeScript and OpenRCT2). The number of instances of all single words, pairs
of words and triples of words were counted for each label (\textit{bug}, \textit{suggestion} and \textit{question}), for each of
the three repositories. All words or phrases that only occurred once were removed, the remaining words and phrases were then
further filtered leaving just the words and phrases that existed with in all three repositories. This was to eliminate specific
terminology only found for a particular repository, e.g. \textit{``JavaScript''} for the TypeScript repository and therefore
leaving a more generalised set of words and phrases that could be used for any repository. Stop words were also removed as they
would act as noise during the training stage, due to their abundance in most literature. A list of stop words provided by
ranks.nl \cite{ranks.nl} was used. tf-idf is an equation that can be used to measure how important a word or phrase is based on
its global frequency over all documents and its frequency for a particular document. This could of have been used to further
reduce the list of words to include as features, however removing a set of known general stop words was considered to be enough
to reduce the set as feature selection based on classifier performance will also take place.

\subsection{Feature selection}
Once initial tests with all the features have been run, the number of features can be narrowed down to reduce the size of the
problem and also reduce the error rate by removing irrelevant or noisy features.

%--------------------------------------------------------------------------
% Experiments
%--------------------------------------------------------------------------
\section{Experiments}
\subsection{k-Nearest Neighbour classifier}
The k-Nearest Neighbour (k-NN) classifier requires no learning algorithm to produce a model. Instead it classifies by an example
by testing against every example within the training data. k-NN classifiers can provide accurate results but are however very
computationally expensive due the the $O(nd)$ running time where $n$ is the number of examples and $d$ is the number of
features. It classifies an example by measuring the distance between the testing example and every training example using a
distance function, for example Euclidean or Hamming. It then chooses the most common class in the nearest $k$ examples. There
are many techniques that can be used to optimise the algorithm but are not relevant to the objective of this project.

Due to the binary features of whether a word or phrase exists or not in the issue body text, a distance function such as
Euclidean would not be appropriate. Instead the distance can be calculated using the Hamming formula as shown in
Equation~\ref{eq:hamming}. This is the total number of features that are different between two examples. Therefore two identical
examples will produce a distance of 0 and will increase, the more words and phrases that are only present in one example and not
the other.
\begin{equation}
	\label{eq:hamming}
	\mathrm{hamming}(x,x') = \sum^d_{j=1} \delta (x_{j} \neq x'_{j})
\end{equation}
where $\delta$ is the indicator function which returns 0 if the word or phrase is present in both examples or neither examples
and 1 if the word or phrase is only present in one example but not the other.

\subsection{Measuring the performance}
Before attempting to improve the classifier using feature selection, a method of measuring the performance reliably is required
so that any improvement or degrade can is shown between manipulation of the classifier. The performance is measured based on the
number of correct predictions made compared with the number of incorrect predictions made. Splitting up the examples obtained
from the application that was written to extract issues from GitHub into two parts allows one part to be used for training and
the other part to be used for testing. Since the correct labels are known, the predicted labels given to the testing examples
can be compared with the correct labels. The number of incorrect predictions made over the number of total predictions is the
error rate.

The process can be performed multiple times where the data is shuffled before each case to reduce the uncertainty and obtain
standard deviation values.

\subsubsection{Cross validation}
Cross validation involves splitting the examples into multiple folds so that each fold can be used for testing and the rest for
training. The final error rates can be averaged to generalise the classifier's performance more as it takes into account
multiple sets of training data and testing data in isolation.

This can be further improved by a technique known as Leave One Out Cross Validation (LOOCV) which involves performing cross
validation on all but one of the folds the data is split into. The folds that give the lowest error rate when used as training
data is then used as the training data for the fold that was not used in the cross validation. The error rate produced by this
final test is then used as the measure of performance.

\subsubsection{Receiver operating characteristics}
\todo{Receiver operating characteristics (ROC)}


% Initial results
Without any feature selection, an average error rate across all labels using cross validation was 26\%. The error rate for just
the \textbf{bug} label was significantly higher at 36\%.

\subsection{Feature selection}
The k-NN classifier was modified to apply feature selection. This was achieved by allowing each feature to have an independent
weight which affected the distance between two examples. Each feature selection technique assigns a weight to each feature where
$w_f = 0$ is a feature that is ignored. This distance function is revised in equation~\ref{eq:hamming_weights}. This will
produce higher distances for features that correlate more.
\begin{equation}
	\label{eq:hamming_weights}
	\mathrm{hamming}(x,x') = \sum^d_{j=1} w_{j} \delta (x_{j} \neq x'_{j})
\end{equation}
where $w_{j}$ is the weight of the feature.

\subsubsection{Mutual Information}
Entropy shows us the measurement of information for a set of data or in other words, the ``amount of randomness''. Entropy can
be used to measure the amount of information gain for a particular feature. By using equation~\ref{eq:entropy}, where $X$ is
the set of feature values or label values, the amount information in the probability distribution of the set can be calculated.
\begin{align}
	\label{eq:entropy}
	H(X) = -\sum_{x \in X} p(x) \log p(x)
\end{align}

Because this scenario only contains binary features (non-existent and existent words / phrases) and binary classes
(e.g. bug or not bug), equation~\ref{binary_entropy} can be used instead which derives from equation~\ref{eq:entropy} but with
the probability and compliment substituted for the summation.
\begin{equation}
	\label{eq:binary_entropy}
	\begin{split}
		H(X) &= -(p(x) \log p(x) + p(x^c)     \log(p(x^c))   \\
		H(X) &= -(p(x) \log p(x) + (1 - p(x)) \log(1 - p(x))
	\end{split}
\end{equation}

To determine whether a feature correlates to a particular class such as \textbf{bug}, the mutual information can be calculated
using equation~\ref{eq:mutual_information}, where $H(Y)$ is the entropy of the class and $H(Y|X)$ is the entropy of the class
given the feature. This will produce numbers within the range $0 \leq I(X;Y) \leq \min(H(X), H(Y))$. The higher the value, the
more the feature correlates to the class, either when it exists or doesn't exist. Each feature can then have an independent
weight based on the mutual information for a particular class or by its rank within the set of features.
\begin{equation}
	\label{eq:mutual_information}
	I(X;Y) = H(Y) - H(Y|X)
\end{equation}

Figure~\ref{fig:mi_prob} shows the probability, $p(\mathrm{bug}|f)$ and mutual information, $I(f,\mathrm{bug})$ for each feature
where the mutual information is above 0.13 bits. The features are ordered from lowest probability to highest probability showing
the least common words / phrases found in issues labelled \textbf{bug} and the most common words / phrases found in issues
labelled \textbf{bug}. The closer the probability is to 0.5, the lower the mutual information, therefore the words / phrases
with very low probability and very high probability will have a high weight as they correlate the most to the label
\textbf{bug}.
\begin{figure}[h]
	\centering
	\includegraphics[width=\linewidth]{charts/mi_prob.pdf}
	\label{fig:mi_prob}
	\caption{Feature probability and mutual information distribution.}
\end{figure}

After calculating the mutual information for each feature and using it for the feature's weight that is applied in the k-NN
distance function as shown in Equation~\ref{eq:hamming_weights}. The error rate for the label \textbf{bug} lowered to 33\%. The
feature selection technique had surprisingly made little difference in improving the accuracy. The experiment was then slightly
modified again so that any feature with a mutual information of less than 0.1 bits was ignored. This lowered the number of
features from 93 to just 34. The error rate was 36\%, the same accuracy prior to feature selection. This meant that over half of
the features were completely irrelevant for the label \textbf{bug}. However when ignoring all features with a mutual information
of less than 0.5 bits, there were only 11 features remaining which produced an error rate of 41\%.

\subsubsection{Conditional Mutual Information}
\todo{}
\cite{fast_binary_feature_selection}

\subsubsection{The `Wrapper' method}
\todo{}

\subsubsection{Filter methods}
\todo{}

%--------------------------------------------------------------------------
% Analysis
%--------------------------------------------------------------------------
\section{Analysis}
\todo{}
\cite{stability_feature_ranking}
\cite{redundant_feature_elimination}

\cite{mltechniques_spamfiltering}
\cite{mlmethods_spamfiltering}

%--------------------------------------------------------------------------
% Conclusions
%--------------------------------------------------------------------------
\section{Conclusions}
\todo{}

%--------------------------------------------------------------------------
% Notes / ideas
%--------------------------------------------------------------------------
% - http://ats.cs.ut.ee/u/kt/hw/spam/spam.pdf page 67
%   http://airccse.org/journal/jcsit/0211ijcsit12.pdf page 176
%   l/k-rule (interesting?)
% 
% - http://computerresearch.org/stpr/index.php/gjcst/article/view/741/650
%   http://www.cs.rit.edu/~nan2563/feature_selection.pdf
%   http://www-nlp.stanford.edu/IR-book/
%   for feature selection techniques
% 
% 
% 
% 