%--------------------------------------------------------------------------
% COMP61011, project
% Edward John
% University of Manchester
%--------------------------------------------------------------------------
\definecolor{light-gray}{gray}{0.9}
\newcommand{\todo}[1]{
	\colorbox{light-gray}{
		\parbox{\linewidth}{
			\textbf{TODO} #1
		}
	}
}

%--------------------------------------------------------------------------
% Abstract
%--------------------------------------------------------------------------
\begin{abstract} 
Issue tracking in software development usually requires the submission of text explaining the issue. Issues tend to have labels
associated with them to help categorise them so that they can be more easily searched for. Examples of labels could include
\textit{bug}, \textit{suggestion} or \textit{question} and may not be mutually exclusive. GitHub is a website that provides
issue tracking to public code repositories. This allows any GitHub user to submit issues wwhere repository collaborators can
then label them.

This projects explores how common labels in particular (\textit{bug}, \textit{suggestion} and \textit{question}) can be
predicted for GitHub issues using machine learning classifiers such as the k-NN classifier and then using feature selection to
improve the accuracy. By using words and phrases in the issue body text as features, the performance can be measured and
compared to determine which words and phrases are more important than others as to what label the issue should be classified as.
\end{abstract}

%--------------------------------------------------------------------------
% Introduction
%--------------------------------------------------------------------------
\section{Introduction}
GitHub is a website of the US company GitHub, Inc. that offers online hosting of code repositories and support for project team
collaboration. One of the provided features for each hosted "Git" repository is an issue tracker. The issue tracker allows users
to submit issues. Types of issues can include; bug reports, feature suggestions, improvements or questions. When an issue is
submitted, collaborators of the repository can add labels to it to help categorise the issue. Labels may not mutually exclusive,
therefore an issue could have the labels; bug and high priority. Because labels have to be manually assigned by collaborators,
issues may not be labelled for some time depending on the availability and number of collaborators. As a result filtering and
sorting may not select or collate all the related issues of a particular type, and collaborators have to be relied upon to be
diligent in keeping up to date with their labelling new issues.

Similar to spam e-mail classifiers, an issue can be classified by analysing the words and phrases in the issue body text. Doing
this requires selecting the right word and phrases (features) to look for. This helps remove noise such as stop words which are
found in nearly all bodies of text for anything and improves the performance of the algorithm by reducing the search space.
Feature selection can also be used to improve the classifiers individually for the labels \textit{bug}, \textit{suggestion} and
\textit{question} by giving each feature a different weight counting towards each label classification.

%--------------------------------------------------------------------------
% Background
%--------------------------------------------------------------------------
\section{Background}
\subsection{Obtaining the data}
\todo{How I obtained the data using GitHub API.}

\subsection{Word / phrase frequencies}
\todo{}

%--------------------------------------------------------------------------
% Experiments
%--------------------------------------------------------------------------
\section{Experiments}
\subsection{k-NN classifier}
\todo{}

\begin{equation}
\mathrm{hamming}(x,x') = \sum^d_{j=1} \delta (x_{j} \neq x'_{j})
\end{equation}

\subsection{Measuring the performance}
\todo{
	\begin{itemize}
		\item Cross validation
		\item Receiver operating characteristics (ROC).
	\end{itemize}
}

\subsection{Feature selection}
\todo{}

%--------------------------------------------------------------------------
% Analysis
%--------------------------------------------------------------------------
\section{Analysis}
Analysis text...

%--------------------------------------------------------------------------
% Conclusions
%--------------------------------------------------------------------------
\section{Conclusions}
Conclusions text...